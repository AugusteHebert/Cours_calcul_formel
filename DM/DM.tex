
\documentclass[11pt,a4paper]{article}
\usepackage[utf8]{inputenc}
\usepackage[T1]{fontenc}
\usepackage[french]{babel}
\usepackage[top=3cm, bottom=2cm, left=2cm, right=2cm]{geometry}
\usepackage{stmaryrd}
\usepackage{amsmath}
\usepackage{amsfonts}
\usepackage{amssymb}
\usepackage{mathrsfs}
\usepackage{amsthm}
\usepackage{layout}
\usepackage{fancyhdr}

\newtheorem*{thm}{Théorème}
\newtheorem{ex}{Exercice}
\newtheorem*{nota}{Notation}
\newtheorem*{rem}{Remarque}
\newtheorem*{rem2}{Remarques}
\newtheorem{de2}{Définition}
\newtheorem{pro2}[de2]{Propriété}
\newtheorem{thm2}[de2]{Théorème}

\setlength{\parindent}{0cm}
\setlength{\parskip}{1ex plus 0.5ex minus 0.2ex}
\newcommand{\hsp}{\hspace{20pt}}
\newcommand{\HRule}{\rule{\linewidth}{0.5mm}}

\usepackage{comment}

\title{}

\date{}
\begin{document}


\pagestyle{fancy}

\fancyhead{}
 \fancyfoot{}

 \lhead{ 2020/2021 \\  L3 Mathématiques
}
\chead{\textbf{ Calcul formel}\\} 
 \rhead{ Université de Lorraine \\  }

\newcommand{\lb}{\llbracket}
\newcommand{\rb}{\rrbracket}
\newcommand{\N}{\mathbb{N}}
\newcommand{\Z}{\mathbb{Z}}




\newcommand{\md}[3]{#1\ \equiv \ #2 \! \! \! \! \! \pmod {#3} }
\newcommand{\nmd}[3]{#1 \not \equiv #2 \! \! \! \! \!  \pmod {#3} }
\newcommand{\mda}[3]{#1 \equiv #2 \! \!  \pmod {#3} }
\newcommand{\nmda}[3]{#1 \not \equiv #2 \! \! \pmod {#3} }
\newcommand{\mo}[2]{#1 \! \! \! \! \! \pmod #2 }
\newcommand{\moa}[2]{#1 \! \!  \pmod {#2} }

\thispagestyle{fancy}

\begin{center}
%    \HRule \\[0.6cm]
    { \huge \bfseries
    DM 
     \\ [0cm] }
    \HRule \\[0.5cm]
\end{center}






\begin{center}
\textbf{A rendre le 14 octobre}
\end{center}

Le but de ce devoir maison est d'étudier les racines carrées de $\overline{1}$ dans $\Z/n\Z$, pour $n\in \N^*$. On commence par traiter un exemple, puis on traite le cas général. Les questions 9 à 11 sont en bonus.

Soit $\phi:\Z/115\Z\rightarrow \Z/5\Z\times \Z/23\Z$ l'isomorphisme chinois (c'est à dire que $\phi(a\mathrm{\ mod\ }115)=(a\mathrm{\ mod\ }5, a\mathrm{\ mod\ }23)$, pour tout $a\in \Z$).

\begin{enumerate}
\item\label{phiMoinsUn_Un} Déterminer $\phi^{-1}\big((1\mathrm{\ mod\ }5,1\mathrm{\ mod\ }23)\big)$ et $\phi^{-1}\big((-1\mathrm{\ mod\ }5,-1\mathrm{\ mod\ }23)\big)$.

\item\label{phiMoinsUn_Un_moinUn} En utilisant les deux algorithmes vus en cours, déterminer $\phi^{-1}\big((1\mathrm{\ mod\ }5,-1\mathrm{\ mod\ }23\big)$. Déterminer $\phi^{-1}\big((-1\mathrm{\ mod\ }5,1\mathrm{\ mod\ }23)\big)$.

\item\label{Racine_carrées_phiMoinsUn} Montrer que $\{x\in \Z/115\Z| x^2=\overline{1}\}=\phi^{-1}(\{x\in \Z/5\Z| x^2=\overline{1}\}\times  \{x\in \Z/23\Z|x^2=\overline{1}\})$. 

\item\label{Carré_Fp} Soit $p\in \mathbb{P}$. En utilisant le fait que $\Z/p\Z$ est intègre, montrer que $\{x\in \Z/p\Z|x^2=\overline{1}\}=\{\overline{-1},\overline{1}\}$. En déduire $\{x\in \Z/115\Z|x^2=\overline{1}\}$. 


\item\label{Question_integrité} Soit  $k\in \N^*$. À quelle condition sur $k$ l'anneau $\Z/p^k\Z$ est-il intègre ?

\item\label{itPGCD} Soit $x\in \Z$. Montrer que $(x+1)\wedge (x-1) $ divise $2$.

\item\label{Determination_Pgcd}  Déterminer $\{x\in \Z|\ (x+1)\wedge (x-1)=1\}$ et $\{x\in \Z|\ (x+1)\wedge (x-1)=2\}$.

\item\label{Racine_carrée_pk} On suppose que  $p \geq 3$. Montrer que si $p^k$ divise $x^2-1$, alors $p^k$ divise $x-1$ ou $x+1$. En déduire $\{x\in  \Z/p^k\Z|x^2=\overline{1}\}$ (on pourra utiliser la question~\ref{itPGCD} et décomposer $x-1$ et $x+1$ en produits de facteurs premiers).

\item\label{Division_2k} Soit $x\in \Z$. On suppose que $2^k$ divise $x^2-1$. Montrer que $2^{k-1}$ divise $x-1$ ou $x+1$ (on pourra utiliser la question~\ref{itPGCD}).

 \item\label{Racine_Carrée_2k} En déduire que si $k\geq 2$, $\{x\in \Z/2^k\Z| x^2=1\}=\overline{1}\}=\{\overline{1}, \overline{2^{k-1}-1},  \overline{2^{k-1}+1}, \overline{-1}=\overline{2^k-1}\}$. Quel est le cardinal de cet ensemble.
 
 \item\label{Racine_carrée_cas_général} Soit $n\in \N_{\geq 2}$. On décompose $n$ en produit de facteurs premiers : $n=2^\alpha p_1^{\beta_1}\ldots p_\ell^{\beta_\ell}$, avec $\alpha\in \N$, $\ell\in \N$, $\beta_1,\ldots,\beta_\ell\in \N^*$,  $3\leq  p_1<\ldots<p_\ell$ et $p_1,\ldots,p_\ell\in \mathbb{P}$. Déterminer $|\{x\in \Z/n\Z|x^2=\overline{1}\}|$  en fonction des valeurs de $\alpha$ et de $\ell$ (on pourra utiliser le théorème chinois). 

\end{enumerate}

\begin{comment}

\paragraph{Correction}

\eqref{phiMoinsUn_Un} On a $\phi\big((1\mathrm{\ mod\ }115)\big)=(1\mathrm{\ mod\ }5,1\mathrm{\ mod\ }23)$ donc $\phi^{-1}\big((1\mathrm{\ mod\ }5,1\mathrm{\ mod\ }23)\big)=1\mathrm{\ mod\ }115$. Comme $\phi^{-1}$ est un morphisme d'anneaux, on a \[\phi^{-1}\big(-(1\mathrm{\ mod\ }5,1\mathrm{\ mod\ }23)\big)=-\phi^{-1}\big((1\mathrm{\ mod\ }5,1\mathrm{\ mod\ }23)\big)=-1\mathrm{\ mod\ }115.\]

\eqref{phiMoinsUn_Un_moinUn} On a $23\times 2-9\times 5=1$, donc $\phi(46\mathrm{\ mod\ }115)=(1\mathrm{\ mod\ } 5,0\mathrm{\ mod\ })$ et $\phi(-45\mathrm{\ mod\ }115)=(0\mathrm{\ mod\ }5,1\mathrm{\ mod\ }23)$. Ainsi $\phi^{-1}\big((1\mathrm{\ mod\ }5,0 \mathrm{\ mod \ }23)\big)=46\mathrm{\ mod\ }115$ et $\phi^{-1}\big((0\mathrm{\ mod\ }5,1 \mathrm{\ mod \ }23)\big)=-45\mathrm{\ mod\ }115=70\mathrm{\ mod\ }115$. On a donc $\phi^{-1}\big((1\mathrm{\ mod\ }5,-1\mathrm{\ mod\ }23)\big)=46-(-45)\mathrm{\ mod\ }115=91\mathrm{\ mod\ }115$.

Cherchons maintenant $\phi^{-1}\big((1\mathrm{\ mod\ }5,-1\mathrm{\ mod\ }23)\big)$ à l'aide de l'algorithme de Garner. Soit $x\in \llbracket 0,114\rrbracket$ tel que $\phi(x\mathrm{\ mod\ }115)=(1\mathrm{\ mod\ }5,-1\mathrm{\ mod\ }23)$. On cherche $x$ sous la forme $n_1+23 n_2$, où $n_1\in \llbracket 0,22\rrbracket$ et $n_2\in \llbracket 0,4\rrbracket$. On a alors $x\equiv -1\equiv 22[23]$, donc $n_1=22$. On a $x\equiv 1[5]$, donc $22+23n_1\equiv 1[5]$, d'où $23n_1\equiv 4[5]$ et donc $3 n_1\equiv 4[5]$. De plus dans $\Z/5\Z$, $\overline{3}\times \overline{2}=\overline{1}$, donc $\overline{n_1}=\overline{4}(\overline{3})^{-1}=\overline{8}$, donc $n_2=3$. On a donc $x=22+3\times 23=91$.  On retrouve que $\phi^{-1}\big((1\mathrm{\ mod\ }5,-1\mathrm{\ mod\ }23)\big)=91\mathrm{\ mod\ }115$. (On aurait aussi pu chercher $x$ sous la forme $n_1+5n_2$, avec $n_2\in  \llbracket 0,22\rrbracket$, mais les calculs auraient été légèrement plus compliqués, car il aurait fallu inverser un élément de $\Z/23\Z$ et non un élément de $\Z/5\Z$.) 

On a $(-1\mathrm{\ mod\ }5,1\mathrm{\ mod\ }23)=-(1\mathrm{\ mod\ }5,-1\mathrm{\ mod\ }23)$ donc $\phi^{-1}\big((-1\mathrm{\ mod\ }5,1\mathrm{\ mod\ }23)\big)=-91 \mathrm{\ mod\ }115=24\mathrm{\ mod\ }115$.

\eqref{Racine_carrées_phiMoinsUn}  Soit $\overline{x}\in \Z/115\Z$.  Comme $\phi$ est un isomorphisme, $\phi$ est bijectif et  $\overline{x}^2=\overline{1}$ si et seulement $\phi(\overline{x}^2)=\phi(\overline{1})$. Donc $\overline{x}^2=\overline{1}$ si et seulement si $(x^2\mathrm{\ mod\ }5,x^2\mathrm{\ mod\ }23)$ si et seulement si $\phi(\overline{x})\in \{y\in \Z/5\Z| y^2=\overline{1}\}\times  \{y\in \Z/23\Z|y^2=\overline{1}\}$.

\eqref{Carré_Fp} Soit $x\in \Z/p\Z$ tel que $x^2=\overline{1}$. Alors $(x^2-\overline{1})=0=(x-\overline{1})(x+\overline{1})=0$, donc $x\in \{\overline{1},\overline{-1})$. On en déduit que \[\begin{aligned}\{x\in \Z/115\Z|x^2=\overline{1}\} &=\phi^{-1}\big(\{1\mathrm{\ mod\ }5,-1\mathrm{\ mod\ }5\}\times\{1\mathrm{\ mod\ }23,-1\mathrm{\ mod\ }23\}\big)\\ &=\phi^{-1}(\{\pm(1\mathrm{\ mod\ }5,1\mathrm{\ mod\ }23),\pm(1\mathrm{\ mod\ }5,-1\mathrm{\ mod\ }23)\}\big)\\
&= \{\pm 1\mathrm{\ mod\ }115,\pm 91\mathrm{\ mod\ }115\}\end{aligned}.\]

\eqref{Question_integrité} Si $k\geq 2$, $\overline{p}\overline{p}^{k-1}=\overline{0}$ donc $\Z/p^k\Z$ n'est pas intègre. Si $k=1$, on  a vu en cours que $\Z/p\Z$ est intègre.

\eqref{itPGCD} Soit $d=(x+1)\wedge (x-1)$. Alors $d$ divise $x+1-(x-1)=2$.

\eqref{Determination_Pgcd} Soit $x\in \Z$. Alors $(x+1)\wedge (x-1)=2$ si et seulement si $x+1$ et $x-1$ sont divisibles par $2$, si et seulement si $x$ est impair. Donc $\{x\in \Z|\ (x+1)\wedge (x-1)=2\}$ est l'ensemble des nombres impairs et $\{x\in \Z|\ (x+1)\wedge (x-1)=1\}$ est l'ensemble des nombres pairs.


\eqref{Racine_carrée_pk} Écrivons les décompositions de $x+1$ et de $x-1$ en produits de facteurs premiers. On a $x+1=\epsilon p_1^{\alpha_1}\ldots p_\ell^{\alpha_\ell}$ et $x-1=\epsilon' p_1^{\alpha_1'}\ldots p_\ell^{\alpha_\ell'}$ où $\epsilon,\epsilon'\in \{-1,1\}$, $p_1,\ldots,p_\ell$ sont des nombres premiers distincts, $\alpha_1,\ldots,\alpha_\ell,\alpha_1',\ldots,\alpha_\ell'\in \N$ et $p_1=p$. Alors on a vu en cours que $(x+1)\wedge (x-1)=p_1^{\beta_1}\ldots p_\ell^{\beta_\ell}$, où $\beta_i=\min (\alpha_i,\alpha'_i)$, pour tout $i\in \llbracket 1,\ell\rrbracket$. Par la question~\ref{itPGCD}, on en déduit que $\beta_1=0$, donc $\alpha_1$ ou $\alpha_1'$ est nul. De plus, $p^k$ divise $(x-1)(x+1)$, donc $\alpha_1+\alpha_1'\geq k$. On en déduit que $\alpha_1\geq k$ ou que $\alpha_1'\geq k$, c'est à dire que $p^k$ divise $x+1$ ou $p^k$ divise $x-1$.  

Soit maintenant $\overline{x}\in \Z/p^k\Z$ tel que $\overline{x}^2=\overline{1}$. Alors $p^k$ divise $x^2-1$ donc $p^k$ divise $x-1$ ou $p^k$ divise $x+1$, donc $\overline{x+1}=\overline{0}$ ou $\overline{x-1}=\overline{0}$. Comme $\overline{1}^2=\overline{1}$ et $\overline{-1}^2=\overline{1}$, on en déduit que $\{\overline{x}\in \Z/p^k\Z|\overline{x}^2=\overline{1}\}=\{\overline{1},\overline{-1}\}$. 

\eqref{Division_2k} On suppose que $2^k$ divise $x^2-1=(x-1)(x+1)$. Écrivons $x+1$ et $x-1$ comme dans la question précédente, avec $p_1=2$. Alors comme $(x+1)\wedge (x-1)|2$, on a $\beta_1\leq 1$. On en déduit que $\alpha_1$ ou $\alpha_1'\leq 1$. Comme $2^k$ divise $(x-1)(x+1)$, $\alpha_1+\alpha_1'\geq k$ et donc $\alpha_1\geq k-1$ ou $\alpha_1'\geq k-1$ ou de manière équivalente $2^{k-1}$ divise $x-1$ ou $x+1$.



\eqref{Racine_Carrée_2k} Soit  $\overline{x}\in \Z/2^k\Z$ tel que $\overline{x}^2=\overline{1}$. On peut supposer que $x\in \llbracket 0,2^k-1\rrbracket$.  D'après la question précédente, $2^{k-1}$ divise $x-1$ ou $x+1$. Supposons   que $2^{k-1}$ divise $x+1$. Alors on peut écrire $x+1=2^{k-1} a$, où $a\in \N^*$. Supposons que $a\geq 3$. Alors $x+1\geq 3\times 2^{k-1}=2^k+2^{k-1}$, donc $x\geq 2^k+2^{k-1}-1\geq 2^k$ : c'est absurde, donc $a\in \{1,2\}$, donc $x\in \{2^{k-1}-1,2^k-1\}$. Supposons que $2^{k-1}$ divise $x-1$. Alors on peut écrire $x-1=2^{k-1}a$, où $a\in \N$. De même que précédemment, on montre que $a\leq 1$. On en déduit que $x\in \{ \Z/2^k\Z|\ x^2=1\}=\{\overline{1}, \overline{2^{k-1}-1}, \overline{2^{k-1}+1},\overline{2^k-1}=\overline{1}\}$. Cet ensemble a $4$ éléments.

\eqref{Racine_carrée_cas_général} Soit $\phi:\Z/n\Z\rightarrow \Z/2^\alpha\Z\times \ldots \Z/p_\ell^{\beta_\ell}\Z$ l'isomorphisme chinois. Alors \[E:=\{x\in \Z/n\Z|x^2=\overline{1}\}=\phi^{-1}(\{x\in \Z/2^\alpha\Z| x^2=\overline{1}\}\times\ldots \{x\in \Z/p_\ell^{\beta_\ell}\Z|x^2=1\}).\] Si $i\in \llbracket 1,\ell\rrbracket$, on a vu que $|\{x\in \Z/p_i^{\beta_i}\Z|x^2=1\}|=2$ et $|\{x\in \Z/2^\alpha\Z| x^2=\overline{1}\}\}|=a_\alpha$ où $a_\alpha$ vaut $1$ si $\alpha\leq 1$, $4$ si $\alpha\geq 2$. On a donc $|E|=a_\alpha 2^\ell$. 


\end{comment}


\end{document}
