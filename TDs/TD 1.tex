
\documentclass[11pt,a4paper]{article}
\usepackage[utf8]{inputenc}
\usepackage[T1]{fontenc}
\usepackage[french]{babel}
\usepackage[top=3cm, bottom=2cm, left=2cm, right=2cm]{geometry}
\usepackage{stmaryrd}
\usepackage{amsmath}
\usepackage{amsfonts}
\usepackage{amssymb}
\usepackage{mathrsfs}
\usepackage{amsthm}
\usepackage{layout}
\usepackage{fancyhdr}
\usepackage{comment}

\newtheorem*{thm}{Théorème}
\newtheorem{ex}{Exercice}
\newtheorem*{nota}{Notation}
\newtheorem*{remarque}{Remarque}
\newtheorem*{remarques}{Remarques}
\newtheorem*{rem}{Remarque}
\newtheorem*{rem2}{Remarques}
\newtheorem{de2}{Définition}
\newtheorem{pro2}[de2]{Propriété}
\newtheorem{thm2}[de2]{Théorème}

\setlength{\parindent}{0cm}
\setlength{\parskip}{1ex plus 0.5ex minus 0.2ex}
\newcommand{\hsp}{\hspace{20pt}}
\newcommand{\HRule}{\rule{\linewidth}{0.5mm}}





\newcommand{\N}{\mathbb{N}}
\newcommand{\R}{\mathbb{R}}
\newcommand{\Z}{\mathbb{Z}}

\title{}

\date{}
\begin{document}


\pagestyle{fancy}

\fancyhead{}
 \fancyfoot{}

 \lhead{ 2020/2021 \\  L3 Mathématiques
}
\chead{\textbf{ Calcul formel}\\} 
 \rhead{   Université de Lorraine \\ }

\newcommand{\lb}{\llbracket}
\newcommand{\rb}{\rrbracket}


\newcommand{\md}[3]{#1\ \equiv \ #2 \! \! \! \! \! \pmod {#3} }
\newcommand{\nmd}[3]{#1 \not \equiv #2 \! \! \! \! \!  \pmod {#3} }
\newcommand{\mda}[3]{#1 \equiv #2 \! \!  \pmod {#3} }
\newcommand{\nmda}[3]{#1 \not \equiv #2 \! \! \pmod {#3} }
\newcommand{\mo}[2]{#1 \! \! \! \! \! \pmod #2 }
\newcommand{\moa}[2]{#1 \! \!  \pmod #2 }


\thispagestyle{fancy}

\begin{center}
%    \HRule \\[0.6cm]
    { \huge \bfseries
    Feuille de TD n$^{\boldsymbol{\circ}}$1
     \\ [0cm] }
    \HRule \\[0.5cm]
\end{center}
\



\begin{ex}\label{exDivision_Euclidienne}\
\begin{itemize}
\item[$1.$] Écrire un algorithme, que l'on notera DivEuc$($a,b$)$, qui à partir de deux entiers $(a,b) \in \mathbb{N}\times \mathbb{N}^* $, donne le quotient et le reste de la division euclidienne de $a$ par $b$.
%\item[$2.$] Généraliser à $(a,b)\in \mathbb{Z}\times \mathbb{Z}^*$
\item[$2.$]
Pour $(a,b) \in \mathbb{N}\times \mathbb{N}^*$, on a l'algorithme suivant:

\begin{center}
\begin{tabular}{l}
DE$($a,b$)$\\
$ Q \leftarrow 0$ \\
$R \leftarrow a$ \\
tant que $R\geq b$ \\
\ \ \ {\rm |} $n \leftarrow 0$ \\
\ \ \ {\rm |}   $c \leftarrow 1$ \\
\ \ \ {\rm |}   tant que $10^n b \leq R$ \\ 
\ \ \ {\rm |} \ \ \ {\rm |} $n \leftarrow n+1$\\
\ \ \ {\rm |}     $n \leftarrow n-1$ \\
\ \ \ {\rm |}    tant que $10^{n}bc \leq R$ \\ 
 \ \ \ {\rm |} \ \ \ {\rm |}    $c \leftarrow c+1$\\
 \ \ \ {\rm |}  $c \leftarrow c-1$ \\
\ \ \ {\rm |}     $Q \leftarrow Q+10^{n}c$\\
\ \ \ {\rm |}   $R \leftarrow R-10^{n}bc$\\
renvoyer $(Q,R)$      \\

\end{tabular}
\end{center}
\ \\
Quel résultat renvoie cet algorithme? Justifier. \\
\textit{Indication: Montrer qu'à chaque étape de la boucle indicée par $k$ dans l'algorithme, la relation $a=bQ_k+R_k$ est vraie (ce qu'on appelle un invariant de boucle).}

\end{itemize}  
  
  
\end{ex}

\begin{remarque}
 L'algorithme précédent est celui utilisé par les élèves du primaire pour effectuer des divisions.% Son coût est moins important que celui du premier algorithme.
%Il existe encore un autre algorithme permettant la division euclidienne de $a$ par $b$, appelé méthode binaire (le premier étant appelé méthode naïve et le second la méthode décimale).
\end{remarque}


\begin{ex}\label{exBases de numération}$($Bases de numération$)$

On admet que, si on fixe $b\geqslant 2$ un entier naturel, alors tout entier naturel $n$ peut s'écrire de façon unique sous la forme
$$ n=a_k b^k +a_{k-1}b^{k-1}+ \ldots + a_1 b +a_0= \overline{a_k a_{k-1}\ldots a_1 a_0}^b  $$
où, $ \forall i \in \lb 0; k \rb $, $a_i \in \lb 0; b-1 \rb$. Les entiers $a_i$ sont appelés les chiffres de la base $b$.
\begin{itemize}
\item[\textbf{Algorithmes}] 
\item[$1.$] Donner une méthode pour calculer $a_0$, puis $a_1, \ldots, a_k$.  En déduire un algorithme, que l'on notera $Pdb(b,n)$, qui à partir de l'écriture en base décimale de $n$ donne son écriture en base $b$. 
\item[$2.$] Soit $n \in \mathbb{N}$ défini par son écriture en base $b$:
$$ n=\overline{a_k a_{k-1}\ldots a_1 a_0}^b  $$
Donner une méthode pour calculer l'écriture de cet entier en base décimale. 
\item[\textbf{Applications}](Calculatrices autorisées)
\item[$3.$] Donner l'écriture de $n=146556$ en base $12$.
\item[$4.$] Donner l'écriture décimale de $n=\overline{16356}^7$.
\end{itemize}

\end{ex}





\begin{ex}\
Écrire un algorithme permettant de calculer le pgcd de trois entiers naturels $a$, $b$ et $c$.
\end{ex}



\

\begin{ex}\label{exOptimisation_algo_Euclide}\
Soient $(a,b)\in \N^*\times \N^*$ tels que $a>b$. Soit  $a=bq+r$ avec $0 \leq  r <b$ la divisiion euclidienne de $a$ par $b$. Montrer que $a\wedge b=b\wedge r=b\wedge (b-r)$. En déduire une «~amélioration~» de l'algorithme d'Euclide. Comparer l'algorithme classique avec ce nouvel algorithme pour le calcul de $89\wedge 55$.
\end{ex}

\



\begin{ex}\label{exPGCD_2n-1_2m-1}\
En utilisant l'algorithme d'Euclide, montrer que pour tous entiers $a,m,n \in \mathbb{N}^*$, on a: $$(a^n-1)\wedge(a^m-1)=a^{m\wedge n}-1.$$
\end{ex}

\


\begin{ex}\label{exComplexite_division_euclidienne}(Calcul de la complexité de l'algorithme d'Euclide)

On appelle suite de Fibonacci, notée $(F_n)_{n \in \mathbb{N}^*}$ la suite d'entiers définie par $F_1=F_2=1$ et pour $n\geqslant 3$, $F_{n+2}=F_{n+1}+F_n$. 
\begin{itemize}
\item[$1.$] Montrer par récurrence sur $k \in \mathbb{N}^*$ que si $a> b > 0$ et si Euclide$(a$,$b)$ nécessite $k$ divisions euclidiennes (l'algorithme d'Euclide se fait donc en $k$ étapes), alors $a \geq F_{k+2}$ et $b \geq F_{k+1}$.
\item[$2.$]En déduire le théorème de Lamé, qui énonce que: $\forall k \in \mathbb{N}^*$, $\forall a,b \in \mathbb{N}$, si $b<F_{k+1}$, alors Euclide$(a$,$b)$ utilise  moins de $k$ divisions euclidiennes.


\item[$3.$] Montrer par récurrence sur $k \in \mathbb{N}^*$ que Euclide $(F_{k+2}$, $F_{k+1})$ s'effectue en exactement $k$ divisions euclidiennes.


\item[$4.$] On admet ici le fait que la suite de terme général $F_{n+1}/F_n$ converge vers le nombre d'or $\varphi$ de valeur $(1+\sqrt{5})/2$ et que $F_n \sim \varphi^n/\sqrt{5}$. Montrer alors que le nombre de divisions euclidiennes nécessitées  par Euclide $(a$,$b)$ est $O(\ln b)$.
\end{itemize}
\end{ex}


\begin{rem}
Historiquement, Fibonacci, vers 1200, a introduit la suite qui porte son nom pour calculer le nombre de couples de lapins au mois $n$ sachant qu'à partir du deuxième mois de maturité un couple de lapin donne naissance à un autre couple et qu'on introduit un couple de lapins à la date $0$. 
\end{rem}

\



\begin{ex}\label{exEuclide_etendu_fibonnacci}\
Quel est le résultat de Euclide-étendu$(F_{k+1},F_k)$ ?
\end{ex}

\


\

\begin{ex}\label{exNombres_Fermat}$($Nombres de Fermat$)$
\begin{itemize}
\item[$1.$] Soit $m$ un entier impair. Montrer que $\forall x \in \mathbb{R}$, 
$$ x^m+1=(x+1)(1-x+x^2-x^3+\ldots + x^{m-1}).$$
\item[$2.$] Soit $n \in \mathbb{N}^*$. Montrer que si $2^n+1$ est premier, alors $n$ est une puissance de $2$.%\\
%\textit{Indication: On se servira du $1.$}
\item[$3.$] Pour $n \in \mathbb{N}$, on pose $F_n=2^{2^{n}}+1$. Montrer que les nombres $\{ F_n \}_n$ sont premiers entre eux deux à deux (on pourra montrer que si $x\in \Z$ et $k\in \N^*$, alors $x^k\equiv -x^{k-1} [x+1]$ et utiliser l'identité de Bézout). En déduire une autre  démonstration de l'infinitude des nombres premiers. 
\end{itemize}
\end{ex}

\begin{rem2}\
\begin{itemize}
\item[•]Les $F_n$ sont appelés les nombres de Fermat. Pour $n \in \lb 0;3 \rb $, Fermat a montré que $F_n$ est premier et a conjecturé qu'il en était de même pour $n \geqslant 5$. Cependant, Euler a montré que $F_5=641 \times 6700417$, et on ne connait aujourd'hui aucun autre $F_n$ premier pour $n \geqslant 5$.
\item[•]Ces nombres ont une application en arithmétique modulaire. Ainsi, Gauss a prouvé le résultat suivant, appelé théorème de Gauss-Wantzel: un polygone régulier à $n$ côtés peut être construit à la règle et au compas si et seulement si $n=2^m$ ou le produit d'une puissance de $2$ et de nombres premiers de Fermat distincts. 
\end{itemize} 
\end{rem2}


\


\begin{ex}\label{exPGCD_suite}\
Soit $(u_n)_{n \in \mathbb{N}}$ la suite d'entiers définie par $u_0=14$ et $u_{n+1}=5 u_n -6$. Montrer que la suite $(u_n\wedge u_{n+1})$ est constante. Précisez sa valeur.
\end{ex}



\
\begin{ex}\
Existe-t-il des entiers premiers à $0$ ? Si oui, lesquels ?
\end{ex}


\


\begin{comment}
\begin{ex}\
On suppose qu'on dispose d'un algorithme de test de primalité TestPrim$($n$)$ qui renvoie vrai si $n$ est premier et faux sinon. A partir de cet algorithme, écrire un algorithme PreSui$($n$)$ qui, à partir d'un entier naturel $n$, renvoie le plus petit nombre premier strictement supérieur à $n$.
\end{ex}

\
\end{comment}

\begin{ex}\label{exPGCD_Polynomes}\
Calculer, pour $n \in \mathbb{N}^*$:
\begin{enumerate}
\item $ (n^2+n) \wedge (2n+1)$
\item $(15n^2+8n+6)\wedge (30 n^2+21n+13)$.
\end{enumerate}
\end{ex}
%\


\


\begin{ex}\label{exNombres_Mersenne}$($Nombres de Mersenne$)$
\begin{itemize}
\item[$1.$] Soient $a \geqslant 2$ et $n \geqslant 2$ deux entiers. Montrer que si $a^n-1$ est premier, alors $a=2$ et $n$ est premier. 
\item[$2.$] Montrer que la réciproque est fausse.\\
\textit{Indication: Étudier le cas $n=11$.}
\item[$3.$] Soit $p$ un nombre premier de la forme $4k+3$ où $k \in \mathbb{N}^*$. Montrer que $2^{(p-1)/2}\equiv (-1)^{k+1}[p]$.\\
\textit{Indication: On pourra poser $N=2^{(p-1)/2}\big( (p-1)/2 \big)!$ et donner deux écritures différentes de $\moa{N}{p}$. }
\item[$4.$] Soit $M_p=2^p-1$ où $p$ est un nombre premier. Montrer que si $p$ est de la forme $4k+3$ et si $2p+1$ est premier, alors $M_p$ n'est pas premier.  
\end{itemize}

\end{ex}


\begin{rem}
Il existe des tests pour déterminer si un nombre de Mersenne donné est premier. Le test de Lucas-Lehmer utilise la suite $(L_n)$ définie par récurrence par $L_0=4$ et $L_{n+1}=L_n^2-2$.\\
Pour $n \geqslant 3$, le nombre $M_n= 2^n -1$ est premier si et seulement si $M_n \mid L_{n-2}$. 
\end{rem}


\section*{Correction}

Exercice~\ref{exDivision_Euclidienne}.

(1) \begin{center}
\begin{tabular}{l}
DivEucl$($a,b$)$\\
$ Q \leftarrow 0$ \\
$R \leftarrow a$ \\
tant que $Q \notin \llbracket 0,b-1\rrbracket$ :\\
\ \ \ {\rm  |}$R\leftarrow R-b$\\
\ \ \ {\rm  |}$Q\leftarrow Q+1$\\
renvoyer $(Q,R)$.

\end{tabular}
\end{center}

\begin{itemize}
\item[•] Terminaison de l'algorithme : si à l'étape $k$, $R_k\geq b$, $R_{k+1}=R_k-b<R_k$ donc à chaque passage dans la boucle, $R$ diminiue strictement. Il y a donc au plus $a$ étapes (en fait $\lfloor a/b\rfloor$).

\item[•] Validité du résultat : si $k$ est tel que $R_k,R_{k+1}$ sont définis, alors $b Q_{k+1} +R_{k+1}= b(Q_k+1)+R_k-b=b Q_k+R_k$. Soit $\ell$ la dernière étape. On a donc $bQ_\ell+R_\ell=b Q_1+R_1=a$.
\end{itemize}

(2) \begin{itemize}
\item[•] Terminaison de l'algorithme : on a $n\geq 0$ et $c\geq 1$ donc si à l'étape $k$, $R_k\geq b$, alors $R_{k+1}\leq R_k-10^n bc\leq R_k-b<R_k$, il y a donc au plus $a$ étapes.

\item[•] Validité du résultat : on à $bQ_{k+1}+R_{k+1}=bQ_k+R_k$ ...
\end{itemize}

Exercice~\ref{exNombres_Fermat}

(1) Si $y\in \R$, on a  $(1-y)(1+y+\ldots y^{m-1})=1-y^m$. En l'appliquant à $y=-x$, on en déduit le résultat.

(2) Supposons que $n$ admet un diviseur impair non trivial $k\geq 3$. Écrivons $n=kn'$, où $n'\in \N$. Alors $2^n+1=(2^{n'})^k+1=(2^{n'}+1)(1-2^{n'}+2^{2n'}+\ldots)$. Comme $1<2^{n'}+1<2^n+1$, on en déduit que $2^n+1$ n'est pas premier. Par contraposée, si $2^n+1$ est premier, $n$ est une puissance de $2$.

(3) Soient $m,n\in \N$ tels que $m\neq n$. Quitte à échanger $m$ et $n$, on peut supposer que $m<n$. Écrivons $2^n=2^m k$, où $k\in \N$ et posons $x=2^m$. Alors $x^k=(x+1)x^{k-1}-x^{k-1}$ donc $x^k\equiv -x^{k-1}[x+1]$.  Par récurrence on en déduit que $x^k\equiv (-1)^\ell x^{k-\ell} [x+1]$, pour $\ell\in \llbracket 0,k\rrbracket$. 
Comme $k$ est pair, on a $x^k\equiv 1 [x+1]$. On a donc $F_{n}=2^{2^n}+1=x^k+1\equiv 2[F_m]$. Autrement dit, il existe $a\in \N$  tel que $F_n -a F_m =2$. Par conséquent, $F_m\wedge F_n $ divise $2$. Comme $F_m$ et $F_n$ sont impairs, on en déduit que $F_m\wedge F_n =1$. 

Si $n\in \N$, on note $P_n$ l'ensemble des diviseurs premiers de $F_n$. Alors si $n\in \N^*$, $P_n$ est non vide et les $P_n$ sont tous disjoints, ce qui prouve que $\bigsqcup_{n\in \N^*}  P_n\subset \mathbb{P}$ est infini.


Exercice~\ref{exBases de numération}


(1) $a_0$ est le reste de la division euclidienne de $n$ par $b$, $a_1$ celui de $\frac{1}{b}(n-a_0)$ par $b$, ...

\begin{center}
\begin{tabular}{l}
$Pdb(b,n)$\\
$N\leftarrow n$ \\
$A\leftarrow [\ ]$\\
$a\leftarrow 0$
tant que $N>0$ : \\
\ \ \ {\rm |} $a\leftarrow N \%b$\\
\ \ \ {\rm |} $A\leftarrow [int(A),a]$\\
\ \ \ {\rm |}  $N\leftarrow \frac{1}{b} (N-a)$ \\
renvoyer $A$.



\end{tabular}
\end{center}

(2) On a $n=\sum_{i=0}^k a_i b^k$. 

(3) $146556=\overline{70990}^{12}$.

(4) $\overline{16356}^7=6.1+5.7+3.7^2+6.7^3+1.7^4=4647$.


Exercice~\ref{exPGCD_suite} On a $u_{n+1}-5u_n=-6$ donc $u_n\wedge u_{n+1}$ divise $6$. Soit $n\in \N$. Supposons que $3$ divise $u_{n+1}$. Alors $3$ divise $u_{n+1}+6=5 u_n$. Comme $5\wedge 3=1$, on en déduit que $3$ divise $u_n$. Par récurrence, si $3$ divise $u_n$, alors $3$ divise $u_0$. On en déduit que pour tous $n\in \N$, $3$ ne divise pas $u_n$. Comme $u_n$ est pair pour tous $n$, $u_n\wedge u_{n+1}=2$, pour tout $n\in \N$.

Exercice~\ref{exPGCD_Polynomes}

(1) Soit $d=(2n+1)\wedge n$. Alors $d$ divise $2(n^2+n)-n(2n+1)=n$, donc $d | (2n+1)\wedge n=1$.

(2) Soit $d=(15n^2+8n+6)\wedge (30 n^2+21n+13)$. Alors $d$ divise  $30 n^2+21n+13-2.(15 n^2+8n+6)=5n+1$ donc $d$ divise $(15 n^2+8n+6)\wedge 5n+1$ donc $d$ divise $15 n^2+8n+6-3n(5n+1)=15 n^2+8n+6-15n^2-3n=5n+6$ donc $d$ divise $(5n+6)\wedge(5n+1)$ donc $d$ divise $5$.

Supposons que $5$ divise $15n^2+8n+6$ et $30 n^2+21n+13$. Alors $5$ divise $8n+6$ donc $3n+1$ et $21n+13$ donc $n+3$. Ainsi, $5$ divise $3n+1-3(n+3)=-8$ : c'est absurde donc $d=1$. 

Exercice~\ref{exNombres_Mersenne}

(1) On a $a^n-1=(a-1)\sum_{i=0}^{n-1} a^i$ donc si $a\geq 3$, $a^n-1$ n'est pas premier. Supposons que $n=n_1n_2$ avec $n_1,n_2\geq 2$. Alors $2^{n_1n_2}-1=(2^{n_1})^{n_2}-1$, qui n'est pas premier par la formule du dessus.

(2) $2^{11}-1=2047=23\times 89$.

(3) Soit $N=2^{(p-1)/2} (\frac{p-1}{2})!$. On a $N=2\times 4\times 6\times \ldots \times (p-1)$. 

Dans $\Z/p\Z$, on a :\[\begin{aligned} \overline{N}=\prod_{i=1}^{k+1} \overline{2i}\prod_{i=k+2}^{2k+2} \overline{2i} &=&  \prod_{i=1}^{k+1} \overline{2i} \prod_{i=k+2}^{2k+2} \overline{2i-4k-3}\\
&=& \prod_{i=1}^{k+1} \overline{2i} \prod_{i=k+2}^{2k+2} \overline{2\big(i-2(k+1)\big)-1}\\
&=& \prod_{i=1}^{k+1} \overline{2i} \prod_{i=-k}^{0} \overline{2i-1}\\
&=& \prod_{i=1}^{k+1} \overline{2i}\overline{(-1)^{k+1}} \prod_{0}^{k} \overline{2i+1}\\
&=& (-1)^{k+1} \overline{(2k+2)!}=(-1)^{k+1}\overline{(\frac{p-1}{2})!}.\end{aligned}.\]

Exercice~\ref{exPGCD_2n-1_2m-1}


Par symétrie, on peut supposer que $m<n$. Comme le cas $m=0$ est clair, on suppose que $m\geq 1$.  Soit $n=mq+r$ la division euclidienne de $n$ par $m$, avec $q\in \N^*$ et $r\in \llbracket 0,m-1\rrbracket$. On a $a^{mq}-1=(a^m-1)\sum_{i=0}^{q-1} a^{mi}$. En particulier, $a^m-1$ divise $a^{mq}-1$. De plus, \begin{equation}\label{eqDiv_eucl}
a^{mq+r}-1=a^r (a^{mq}-1)+a^{r}-1=(a^{m}-1)a^{r}\frac{a^{mq}-1}{a^m-1}+a^{r}-1.
\end{equation}  
Comme $r<m$, $a^r-1<a^m-1$ et donc \eqref{eqDiv_eucl} est la division euclidienne de $a^n-1=a^{mq+r}-1$ par $a^m-1$.

Posons $r_0=n$, $r_1=m$. On définit $(r_i)_{i\in \llbracket 1,k+2\rrbracket}$ par récurrence comme suit : tant que $r_{i+1}\neq 0$, on effectue la division euclidienne de $r_{i}$ par $r_{i+1}$ : \[r_i=q_{i+1} r_{i+1}+ r_{i+2},\] où $r_{i+2}\in \llbracket 0, r_{i+1}\llbracket$. Alors si $i\in \llbracket 1,k\rrbracket$, la division euclidienne de $a^{r_i}-1$ par $a^{r_{i+1}}-1$ s'écrit $a^{r_i}=q'_{i+1} (a^{r_{i+1}}-1)+a^{r_{i+2}}-1$, pour un certain $q'_{i+1}\in \Z$. La suite $(a^{r_i}-1)_{i\in \llbracket 0,k+2\rrbracket}$ est donc la suite obtenue en appliquant l'algorithme d'Euclide à $a^{n}-1$ et $a^m-1$, d'où le résultat.

Exercice~\ref{exOptimisation_algo_Euclide}

On a $\Z a+\Z b =\Z b+\Z r=\Z b+ \Z (b-r)$ (double inclusion). De manière alternative :

Le nombre $b\wedge r$ divise $b$ et $r$ donc aussi $b-r$. On a donc $b\wedge r\  |\  b\wedge (b-r)$. De même, $b\wedge (b-r)$ divise $b$ et $b-r$ donc $b-(b-r)=r$, donc $b\wedge (b-r)\ |\  b\wedge r$ et donc  $b\wedge r=b\wedge (b-r)$.


Exercice~\ref{exComplexite_division_euclidienne}

(1) Supposons que l'algorithme d'Euclide se termine en $1$ étape. Alors $a>b\geq 1=F_2$ et $a\geq 2=F_{3}$. Soit $k\in \N^*$. On suppose que le résultat est vrai pour $k$. Soient $a,b$ tels que Euclide$(a,b)$ effectue $k+1$ divisions euclidiennes. Effectuons la division euclidienne de $a$ par $b$ : $a=bq+r$ avec $r\in \llbracket 0,b-1\rrbracket$ et $q\in \Z$. Alors Euclide$(b,r)$ s'effectue en $k$ étapes, donc $b\geq F_{k+2}$ et $r\geq F_{k+1}$. De plus, $a\geq b+r\geq F_{k+2}+F_{k+1}=F_{k+3}$, donc le résultat est vrai pour $k+1$ donc le résultat est vrai par récurrence.

(2) Par contraposée, on en déduit le théorème de Lamé.

(3) La division euclidienne de $F_{i+2}$ par $F_{i+1}$ est $F_{i+2}=F_{i+1}+F_i$, pour tous $i\in \N$. 

(4) On pose $k(a,b)$ le nombre de divisions euclidiennes effectuées par l'algorithme d'Euclide. Soit $n(b)\in \N$ l'unique entier vérifiant $F_{n(b)}\leq n(b)<F_{n(b)+1}$. On a alors $k(a,b)\leq n(b)$. 

Soit $K\in \N$ tel que pour tout $m\geq K$, $F_m\geq \frac{1}{2\sqrt{5}}\varphi^{m}$. Alors $n(b)\rightarrow_{b\rightarrow +\infty} +\infty$ et donc pour $b\gg 0$, $n(b) >K$. On a alors $\frac{1}{2\sqrt{5}} \varphi^{n(b)}\leq b$ donc $n(b)\ln(\varphi)+\ln(\frac{1}{2\sqrt{5}})\leq \ln(b)$, ce qui montre le résultat.

Soit $k\in \N_{\geq 4}$. On suppose que $(-1)^k F_k F_{k-3} +(-1)^{k+1} F_{k-1} F_{k-2}=1$. Alors \[\begin{aligned} 1 &=& (-1)^k F_k F_{k-3} +(-1)^{k+1} F_{k-1} F_{k-2} \\
&=& (-1)^k F_k (F_{k-1}-F_{k-2})+(-1)^{k+1} (F_{k+1}-F_k) F_{k-2}\\
&=& (-1)^{k+1} F_{k+1} F_{k-2}+(-1)^k F_k(F_{k-2}+F_{k-3})\\
1 &=& (-1)^{k+1} F_{k+1} F_{k-2} + (-1)^k F_kF_{k-1}.\end{aligned}\]




\end{document}
